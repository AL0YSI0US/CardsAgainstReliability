% options:
%\def\BLEEDAREA{} % put bleed area around image as required for printing at makeplayingcards.com https://www.makeplayingcards.com/dl/booklet-template/us-game-4pp.pdf
%\def\LINES{} % put lines for cutting

\newcommand{\drawbackground}{
	\ifdefined\BLEEDAREA
	% background to bleed area
	\fill[cardbg] (-0.12in, -0.12in) rectangle +(2.44in, 3.67in);
	\fi
	% background only to cut area
	\fill[cardbg] (0in, 0in) rectangle +(2.2in, 3.43in);

	\ifdefined\LINES
	% guideline - cut area
	\draw[green] (0in, 0in) rectangle +(2.2in, 3.43in);
	% guideline - safe area
	%\draw[red] (0.12in, 0.12in) rectangle +(1.96in, 3.19in);
	\fi
}

\newcommand{\drawcardicon}{
	\node [above right] at (0in + 0.12in, 3.43in - 0.12in) {\cardicon};
}

\newcommand{\drawbodytext}[1]{
	\node [cardfg,below right,text width=1.86in] at (0.12in, 0.12in) {\bodyfont #1};
}

\newcommand{\drawcardtitle}{
	\node [cardfg,below right] at (0.12in, 0.12in) {\titlefont Cards};
	\node [cardfg,below right] at (0.12in, 0.42in) {\titlefont Against};
	\node [cardfg,below right] at (0.12in, 0.72in) {\titlefont Reliability};
}
